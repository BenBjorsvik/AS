\subsection{Ledd og faktorer} 
De fire regneartene:
\begin{align*}
    \text{addisjon: }& &&ledd + ledd = sum \\
    \text{subtraksjon: }& &&ledd - ledd = differanse \\
    \text{multiplikasjon: }& &&faktor \cdot faktor = produkt \\
    \text{divisjon: }& &&dividend / divisor = \frac{teller}{nevner} = kvotient
\end{align*}
\subsection{Brøkregning}
Her vil du få spesielt bruk for primtallsfaktorisering!

Addisjon og subtraksjon: utvidelse av brøker.
Multiplikasjon og divisjon: to sider av samme sak.
Stryking av faktorer er litt som å unngå dobbelt arbeid. Det er unødvendig å doble for så å halvere rett etterpå, og vice versa.

\subsection{Potenser og røtter}
\subsection{Regnerekkefølge - presedensregler} 
\begin{enumerate}
    \item Parenteser
    \item Potenser
    \item Multiplikasjon og divisjon
    \item Addisjon og subtraksjon
\end{enumerate}
\subsection{Variabler og ukjente} 
\subsection{Lineære likninger} 
\subsection{Lineære ulikheter}

